
\section{Combined Uncertainty Decomposition}

This figure shows the decomposition of total prediction uncertainty into its constituent components, 
following the variance decomposition formula:

$$\text{Var}_{\text{total}}[f(x)] = \mathbb{E}_b[\text{Var}_{\theta|b}[f(x|\theta)]] + \text{Var}_b[\mathbb{E}_{\theta|b}[f(x|\theta)]]$$

where $b$ indexes bootstrap samples and $\theta|b$ represents the posterior distribution for bootstrap sample $b$.

\textbf{Uncertainty Components:}

\begin{itemize}
\item \textbf{Model uncertainty} (blue): $\mathbb{E}_b[\sigma_b^2(x)]$ - Average within-bootstrap variance representing 
  uncertainty in our model predictions given a fixed dataset. This captures epistemic uncertainty 
  about the model parameters.

\item \textbf{Data uncertainty} (orange): $\text{Var}_b[\mu_b(x)]$ - Between-bootstrap variance representing 
  uncertainty due to finite sample size. This captures aleatoric uncertainty arising from 
  sampling variability.
\end{itemize}

\textbf{Left panels}: Absolute variance contributions showing how much each uncertainty source 
contributes to the total variance at each $x$ value.

\textbf{Right panels}: Relative contributions showing the fraction of total variance attributable 
to each source, with values summing to 1.

\textbf{Interpretation:}
\begin{itemize}
\item Regions where model uncertainty dominates suggest the model is well-constrained by the data 
  but has inherent parameter uncertainty
\item Regions where data uncertainty dominates suggest more data would significantly reduce uncertainty
\item The relative importance can guide experimental design and model improvement strategies
\end{itemize}

This decomposition is essential for understanding whether uncertainty reduction efforts should 
focus on improving the model or collecting more data.
