
\section{Function-Space (Predictive) Uncertainty}

This figure demonstrates how parameter uncertainty propagates to function predictions, 
showing the predictive uncertainty $p(f(x)|\mathcal{D})$ obtained by marginalizing 
over the parameter posterior:

$$p(f(x)|\mathcal{D}) = \int p(f(x)|\theta) p(\theta|\mathcal{D}) d\theta$$

The visualization includes:

\begin{itemize}
\item \textbf{Uncertainty bands}: Shaded regions showing 50\% (dark) and 90\% (light) 
  confidence intervals for function predictions at each $x$
\item \textbf{Median prediction}: Blue solid line showing the median $f(x)$ across all posterior samples
\item \textbf{True function}: Red dashed line showing the ground truth $f(x|\theta_{\text{true}})$
\item \textbf{Sample functions}: Gray lines showing individual function realizations from posterior samples
\end{itemize}

The computation procedure:
\begin{enumerate}
\item Sample parameters $\{\theta^{(i)}\}$ from the posterior $p(\theta|\mathcal{D})$
\item Evaluate function $f(x|\theta^{(i)})$ for each sample at all $x$ values
\item Compute empirical quantiles across samples to form confidence bands
\end{enumerate}

The width of the uncertainty bands indicates how confident we are in our function 
predictions given the observed data. Wider bands indicate higher uncertainty, 
typically occurring in regions where the data provides less constraint.
